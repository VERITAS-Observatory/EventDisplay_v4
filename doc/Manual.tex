%*-- Author :    G.Maier   2011/09/25
\documentclass[titlepage,a4paper,twoside,11pt]{report}
% floating objects
\usepackage{float}
% floating object placements
\usepackage{afterpage}
% use more of a page
\usepackage{a4wide}
% sort citation numbers
\usepackage{cite}
% more symbols, e.g. \blacksquare, \blacktriangle
\usepackage{amssymb}
% \usepackage{lscape}  % landscape with \begin{landscape} \end{landscape}
% header with lines etc.
\usepackage{fancyheadings}
\usepackage{longtable} % table longer than one page
\usepackage{booktabs} % toprule, etc.

% captions in italic
\usepackage{caption2}
\renewcommand*\captionfont{\slshape}
\renewcommand*\captionlabelfont{\upshape}
% versions
\usepackage{version}
 \includeversion{TEV}
 \excludeversion{GREEN}
 \excludeversion{LMX}
 \excludeversion{HMX}
 \excludeversion{EGRET}
\excludeversion{TAYLOR}

\usepackage[pdftex]{graphicx}
\pdfcompresslevel=0   % pdf-file should (not) be compressed
\usepackage[pdftex]{color}
% make bookmarks, don't show bookmarks when opening the file
\usepackage[pdftex,colorlinks=false,bookmarks=false,bookmarksopen=false]{hyperref}
\pdfinfo
  {
      /Title (promo.pdf)
      /Creator (TeX)
      /Producer (pdfTeX)
      /Author (Gernot Maier)
      /CreationDate (D:20110925180100)
      /ModDate (D:20110925180100)
      /Subject (manual)
      /Keywords (gamma rays, analysis, VERITAS, CTA)
  }
%  \usepackage{thumbpdf}

%%%%%%%%%%%%%%%
%% pdflatex configure file ????
%% To include a graphics file \includegraphics{filename_to_include}
%% convert eps files with epstopdf

% no header and no page number, ersetzt \cleardoublepage
\newcommand{\clearemptydoublepage}{\newpage{\pagestyle{empty}\cleardoublepage}}

% new headers
\pagestyle{fancyplain}
\renewcommand{\chaptermark}[1]{\markboth{#1}{}}
\renewcommand{\sectionmark}[1]{\markright{\thesection\ #1}}
\lhead[\fancyplain{}{\bfseries\thepage}]{\fancyplain{}{\bfseries\rightmark}}
\rhead[\fancyplain{}{\bfseries\leftmark}]{\fancyplain{}{\bfseries\thepage}}
\cfoot{\fancyplain{\thepage}}

% \marginlabel, replaces \marginpar 
\newcommand{\marginlabel}[1]{\mbox{}\marginpar{\raggedleft\hspace{0pt}{\color{cyan}#1}}}
% degrees
\newcommand{\Grad}{$^{\circ}$}
%% new color
\definecolor{Grau}{gray}{0.905}
\usepackage{listings}
\lstset{language=C++, basicstyle=\small,backgroundcolor=\color{Grau}}

\renewcommand{\textfraction}{0.02}
\renewcommand{\floatpagefraction}{0.70}

% spelling
\hyphenation{CRES KRETA KASCADE GHEISHA Karls-ruhe CORSIKA}

\begin{document}

% ########### title page ###########################

%*-- Author :    G.Maier   29/09/2011

\newcommand{\HRule}{\rule{\linewidth}{1mm}}

\begin{titlepage}
\thispagestyle{empty}
\setlength{\parindent}{0mm}
\setlength{\parskip}{0mm}

\vspace*{10mm}
\HRule
\begin{flushright}
 \Huge \bf
  EVNDISP Manual \\
    IACT event analysis and display \\
     v4.00
\end{flushright}
\HRule
%\vspace*{\stretch{0.05}}
\vspace{10mm}
\begin{center}
%   \Huge
%   {\textsl{Gernot Maier}} \\[5mm]
   \Large
   
   \vspace{3.5cm}
  
   \vspace{0.5cm}
    
   \vspace{0.5cm}
   Gernot Maier (DESY) \\
   .....\\

   \vspace{6.3cm}

\end{center}
\begin{flushleft}
\Large
 SVN $Rev$  \\
SVN $Date$ \\
SVN $Author$

\end{flushleft}

%\vspace{10mm}
%\begin{figure}[H]
%\ifpdf
%!TEX encoding = UTF-8 Unicode
%
%\centering\includegraphics[width=\linewidth]{../bilder/P_kascade/pana_kascade1.jpg}
%\else
%\centering\includegraphics[width=\linewidth]{../bilder/P_kascade/pana_kascade1.eps}
%\fi
%\end{figure}

%\vspace*{\stretch{2}}
%\begin{center}
%  \large Version 0.02/04 \ifBW (BW) \else (CO) \fi \today \\
%  \large gernot.maier@ik.fzk.de
%\end{center}
\end{titlepage}


% ############ Vorspann  ###########################

\pagenumbering{roman}
\thispagestyle{empty}

\begin{abstract}

eventdisplay is a complete package for VERITAS and CTA analysis (whatever 'complete' means...)
The package consists of several analysis steps and tools:

\begin{enumerate}
\item evndisp (calibrate and parametrize images, event reconstruction, stereo analysis)
\item mscw\_energy (use lookup tables to produce msw, msl, and energies)
\item anasum (produce maps and calculate analysis results)
\item shared library tools and macros (see EVNDISP/lib/libVAnaSum.so and EVNDISP/macros/)  (produce the energy spectrum and integral fluxes, plot maps, etc.) 
\item makeEffectiveArea (calculate effective areas)
\item makeOptimizeBoxCutsTMVA (tools to optimize cuts)
\item ...
\end{enumerate}

This is a very incomplete manual, started in September 2009. Please help updating it.

\end{abstract}


\setcounter{page}{1}
\tableofcontents
\clearemptydoublepage


% ############ main part #####################
\pagenumbering{arabic}

\chapter{Documentation}

EVNDISP is work-in-progress and the documentation is not in the state it is supposed to be. 
Apart from the information in this manual, other sources for help are:

\subsection*{README files}

\begin{description}
\item[INSTALL:]   information on installation the analysis package, dependencies, environmental variables
\item[README.CTA :]  short description of a typical CTA analysis
\item[README.VTS:]      description of a typical VERITAS analysis
\item[AUTHORS:]		author description
\end{description}

\noindent Description and command line options for the different software parts:

\begin{description}
\item[README.EVNDISP:]
\item[README.MSCW\_ENERGY:]
\item[README.ANASUM:]
\item[README.EFFECTIVEAREA:]
\item[README.ANALYSISLIBRARY:]
\item[README.SCRIPTS:]
\item[README.MACROS:]
\end{description}

\subsection*{WIKI pages}

The EVNDISP manual for VERITAS users: 

\url{http://veritas.sao.arizona.edu/wiki/index.php/Eventdisplay\_Manual}

\clearemptydoublepage

\chapter{Introduction}

\chapter{eventdisplay - calibration, image analysis and stereo reconstruction}

% ############ Input data format ######################
\chapter{Input data format}

\section{VBF - VERITAS Bank Format}

\section{DST - data summary tree}

The DST format is a simple ROOT tree containing standard C++ variables only (no class data).

\subsection{Limitations}

The implementation requires the hardwiring of the maximum number of telescopes, channels, etc. 
These values can be found in {\texttt inc/VGlobalRunParameter}, for example:

\begin{lstlisting}
// HARDWIRED MAXIMUM NUMBER OF TELESCOPES AND CHANNELS, etc.
// maximum number of telescopes
#define VDST_MAXTELESCOPES  100
// maximum number of channels per telescopes
#define VDST_MAXCHANNELS   12000
// maximum number of summation windows
// (=maximum number of samples per FADC trace)   
#define VDST_MAXSUMWINDOW   16
// maximum number of time slices for pedestal calculation
#define VDST_PEDTIMESLICES 5000   
// maximum number of arrayreconstruction method 
#define VDST_MAXRECMETHODS  100
// maximum number of timing levels
#define VDST_MAXTIMINGLEVELS 10    
\end{lstlisting}

\noindent {\bf NOTE: } These numbers determine the memory requirements of {\it evndisp} and {\it CTA.convert\_hessio\_to\_VDST}.

\noindent {\bf NOTE: } {\it evndisp} must be compiled with the same settings as the writing program.

% ############ CTA analysis ######################

\chapter{CTA analysis}

\section{General concept}

To use Eventdisplay for CTA analyses, you first have to convert the {\it simtel.gz} file into ROOT format. 
This you have to do for each subarray separately.
Afterwards you can analyze this subarray file running {\it eventdisplay} in the standard way, which means:

\begin{description}
\item[\$EVNDISPSYS/bin/CTA.convert\_hessio\_to\_VDST]
        Convert simtel.gz files into ROOT format

\item[\$EVNDISPSYS/bin/eventdisplay]
	ImageCleaning \& Calculation of telescope parameters \& Reconstruction of the direction

\item[\$EVNDISPSYS/bin/mscw\_energy]
        Use lookup tables to estimate the energy

\item[\$EVNDISPSYS/bin/makeEffectiveArea]
        Make effective areas

\item[\$EVNDISPSYS/macros/plot\_sensitivity.C]
        plot sensitivity curves
\end{description}

(For most of the stuff there are scripts to run the analysis on the DESY batch system. Use them to simplify your life!)

\section{Analysis}

{\bf NOTE:}
By converting a CTA.simtel.gz file into evndisp.root format you have to specify a subarray!
EVNDISP is not able to handle the complete 275 telescopes at once!

\subsection{Step 1: Converter}

EVNDISPSYS/bin/CTA.convert\_hessio\_to\_VDST

(run it to all the possible inputparameter)

\subsection{Display (event-by-event)}

EVNDISPSYS/bin/evndisp

e.g. 

\$EVNDISPSYS/bin/evndisp -display=1 -highres -useFixedThresholds -imagethresh=10.0 -borderthresh=5.0 -sourcefile \$CTA\_USER\_DATA\_DIR/simtel.root

\subsection{Step 1 \& 2 combined [USE]}

Run the converter and Eventdisplay for a specific subarray: use script \$EVNDISPSYS/scripts/CTA/CTA.EVNDISP.sub\_convert\_and\_analyse\_MC\_VDST.sh

NOTE: The image cleaning thresholds can be specified in the file \$EVNDISPDATA/ParameterFiles/EVNDISP.reconstruction.runparameter file either
for all telescopes to the same values or for each telescope type seperately. 

\subsection{Step 3: mscw\_energy}

If you use standard configurations maybe some lookup tables already exist (ask Gernot or Heike where you could find them).

If not, you have to create them yourself with

\$EVNDISPSYS/scripts/CTA/CTA.MSCW\_ENERGY.sub\_make\_tables.sh

If you have your lookup tables you have to run mscw\_energy for estimating the energy of each event: 

\$EVNDISPSYS/scripts/CTA/CTA.MSCW\_ENERGY.sub\_analyse\_MC.sh

Step 4: Effective Areas
-----------------------

To calculate effective areas, look at:

\$EVNDISPSYS/scripts/CTA/CTA.EFFAREA.sub\_analyse.sh

For the first time you have to use mscw files as input since MCpars has to be analysed once. 
If you change your cuts afterwards (but not the number of input files) you can use the faster version by using
the effective area file as input (helps a lot for protons).



% ############ Appendix ######################

\begin{appendix}
 coordinate systems
\end{appendix}


%\bibliographystyle{plain_gm}
%\bibliographystyle{hep}
%\cleardoublepage
%\addcontentsline{toc}{chapter}{\bibname}
%\bibliography{BibAstro,BibStat,BibPhys,BibPart,BibEtc,BibKASCADE}

% #############################################


\end{document}
